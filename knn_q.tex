% Options for packages loaded elsewhere
\PassOptionsToPackage{unicode}{hyperref}
\PassOptionsToPackage{hyphens}{url}
%
\documentclass[
]{article}
\title{Gov 2018 Lab 1: k nearest neighbors}
\author{Adeline Lo}
\date{January 22, 2022}

\usepackage{amsmath,amssymb}
\usepackage{lmodern}
\usepackage{iftex}
\ifPDFTeX
  \usepackage[T1]{fontenc}
  \usepackage[utf8]{inputenc}
  \usepackage{textcomp} % provide euro and other symbols
\else % if luatex or xetex
  \usepackage{unicode-math}
  \defaultfontfeatures{Scale=MatchLowercase}
  \defaultfontfeatures[\rmfamily]{Ligatures=TeX,Scale=1}
\fi
% Use upquote if available, for straight quotes in verbatim environments
\IfFileExists{upquote.sty}{\usepackage{upquote}}{}
\IfFileExists{microtype.sty}{% use microtype if available
  \usepackage[]{microtype}
  \UseMicrotypeSet[protrusion]{basicmath} % disable protrusion for tt fonts
}{}
\makeatletter
\@ifundefined{KOMAClassName}{% if non-KOMA class
  \IfFileExists{parskip.sty}{%
    \usepackage{parskip}
  }{% else
    \setlength{\parindent}{0pt}
    \setlength{\parskip}{6pt plus 2pt minus 1pt}}
}{% if KOMA class
  \KOMAoptions{parskip=half}}
\makeatother
\usepackage{xcolor}
\IfFileExists{xurl.sty}{\usepackage{xurl}}{} % add URL line breaks if available
\IfFileExists{bookmark.sty}{\usepackage{bookmark}}{\usepackage{hyperref}}
\hypersetup{
  pdftitle={Gov 2018 Lab 1: k nearest neighbors},
  pdfauthor={Adeline Lo},
  hidelinks,
  pdfcreator={LaTeX via pandoc}}
\urlstyle{same} % disable monospaced font for URLs
\usepackage[margin=1in]{geometry}
\usepackage{color}
\usepackage{fancyvrb}
\newcommand{\VerbBar}{|}
\newcommand{\VERB}{\Verb[commandchars=\\\{\}]}
\DefineVerbatimEnvironment{Highlighting}{Verbatim}{commandchars=\\\{\}}
% Add ',fontsize=\small' for more characters per line
\usepackage{framed}
\definecolor{shadecolor}{RGB}{248,248,248}
\newenvironment{Shaded}{\begin{snugshade}}{\end{snugshade}}
\newcommand{\AlertTok}[1]{\textcolor[rgb]{0.94,0.16,0.16}{#1}}
\newcommand{\AnnotationTok}[1]{\textcolor[rgb]{0.56,0.35,0.01}{\textbf{\textit{#1}}}}
\newcommand{\AttributeTok}[1]{\textcolor[rgb]{0.77,0.63,0.00}{#1}}
\newcommand{\BaseNTok}[1]{\textcolor[rgb]{0.00,0.00,0.81}{#1}}
\newcommand{\BuiltInTok}[1]{#1}
\newcommand{\CharTok}[1]{\textcolor[rgb]{0.31,0.60,0.02}{#1}}
\newcommand{\CommentTok}[1]{\textcolor[rgb]{0.56,0.35,0.01}{\textit{#1}}}
\newcommand{\CommentVarTok}[1]{\textcolor[rgb]{0.56,0.35,0.01}{\textbf{\textit{#1}}}}
\newcommand{\ConstantTok}[1]{\textcolor[rgb]{0.00,0.00,0.00}{#1}}
\newcommand{\ControlFlowTok}[1]{\textcolor[rgb]{0.13,0.29,0.53}{\textbf{#1}}}
\newcommand{\DataTypeTok}[1]{\textcolor[rgb]{0.13,0.29,0.53}{#1}}
\newcommand{\DecValTok}[1]{\textcolor[rgb]{0.00,0.00,0.81}{#1}}
\newcommand{\DocumentationTok}[1]{\textcolor[rgb]{0.56,0.35,0.01}{\textbf{\textit{#1}}}}
\newcommand{\ErrorTok}[1]{\textcolor[rgb]{0.64,0.00,0.00}{\textbf{#1}}}
\newcommand{\ExtensionTok}[1]{#1}
\newcommand{\FloatTok}[1]{\textcolor[rgb]{0.00,0.00,0.81}{#1}}
\newcommand{\FunctionTok}[1]{\textcolor[rgb]{0.00,0.00,0.00}{#1}}
\newcommand{\ImportTok}[1]{#1}
\newcommand{\InformationTok}[1]{\textcolor[rgb]{0.56,0.35,0.01}{\textbf{\textit{#1}}}}
\newcommand{\KeywordTok}[1]{\textcolor[rgb]{0.13,0.29,0.53}{\textbf{#1}}}
\newcommand{\NormalTok}[1]{#1}
\newcommand{\OperatorTok}[1]{\textcolor[rgb]{0.81,0.36,0.00}{\textbf{#1}}}
\newcommand{\OtherTok}[1]{\textcolor[rgb]{0.56,0.35,0.01}{#1}}
\newcommand{\PreprocessorTok}[1]{\textcolor[rgb]{0.56,0.35,0.01}{\textit{#1}}}
\newcommand{\RegionMarkerTok}[1]{#1}
\newcommand{\SpecialCharTok}[1]{\textcolor[rgb]{0.00,0.00,0.00}{#1}}
\newcommand{\SpecialStringTok}[1]{\textcolor[rgb]{0.31,0.60,0.02}{#1}}
\newcommand{\StringTok}[1]{\textcolor[rgb]{0.31,0.60,0.02}{#1}}
\newcommand{\VariableTok}[1]{\textcolor[rgb]{0.00,0.00,0.00}{#1}}
\newcommand{\VerbatimStringTok}[1]{\textcolor[rgb]{0.31,0.60,0.02}{#1}}
\newcommand{\WarningTok}[1]{\textcolor[rgb]{0.56,0.35,0.01}{\textbf{\textit{#1}}}}
\usepackage{longtable,booktabs,array}
\usepackage{calc} % for calculating minipage widths
% Correct order of tables after \paragraph or \subparagraph
\usepackage{etoolbox}
\makeatletter
\patchcmd\longtable{\par}{\if@noskipsec\mbox{}\fi\par}{}{}
\makeatother
% Allow footnotes in longtable head/foot
\IfFileExists{footnotehyper.sty}{\usepackage{footnotehyper}}{\usepackage{footnote}}
\makesavenoteenv{longtable}
\usepackage{graphicx}
\makeatletter
\def\maxwidth{\ifdim\Gin@nat@width>\linewidth\linewidth\else\Gin@nat@width\fi}
\def\maxheight{\ifdim\Gin@nat@height>\textheight\textheight\else\Gin@nat@height\fi}
\makeatother
% Scale images if necessary, so that they will not overflow the page
% margins by default, and it is still possible to overwrite the defaults
% using explicit options in \includegraphics[width, height, ...]{}
\setkeys{Gin}{width=\maxwidth,height=\maxheight,keepaspectratio}
% Set default figure placement to htbp
\makeatletter
\def\fps@figure{htbp}
\makeatother
\setlength{\emergencystretch}{3em} % prevent overfull lines
\providecommand{\tightlist}{%
  \setlength{\itemsep}{0pt}\setlength{\parskip}{0pt}}
\setcounter{secnumdepth}{-\maxdimen} % remove section numbering
\ifLuaTeX
  \usepackage{selnolig}  % disable illegal ligatures
\fi

\begin{document}
\maketitle

This exercise is based on:

Pager, Devah. (2003). ``\href{https://doi.org/10.1086/374403}{The Mark
of a Criminal Record}.'' \emph{American Journal of Sociology}
108(5):937-975. You are also welcome to watch Professor Pager discuss
the design and result \href{https://youtu.be/nUZqvsF_Wt0}{here}.

To isolate the causal effect of a criminal record for black and white
applicants, Pager ran an audit experiment. In this type of experiment,
researchers present two similar people that differ only according to one
trait thought to be the source of discrimination. We will be using this
dataset to see if certain applicant characteristics (including race) can
help us predict out of sample callback rates for job applications.

In the data you will use \texttt{criminalrecord.csv} nearly all these
cases are present, but 4 cases have been redacted. We've kept these
unnecessary variables in the dataset because it is common to receive a
dataset with much more information than you need.

\begin{longtable}[]{@{}
  >{\raggedright\arraybackslash}p{(\columnwidth - 2\tabcolsep) * \real{0.31}}
  >{\raggedright\arraybackslash}p{(\columnwidth - 2\tabcolsep) * \real{0.69}}@{}}
\toprule
\begin{minipage}[b]{\linewidth}\raggedright
Name
\end{minipage} & \begin{minipage}[b]{\linewidth}\raggedright
Description
\end{minipage} \\
\midrule
\endhead
\texttt{jobid} & Job ID number \\
\texttt{callback} & 1 if tester received a callback, 0 if the tester did
not receive a callback. \\
\texttt{black} & 1 if the tester is black, 0 if the tester is white. \\
\texttt{crimrec} & 1 if the tester has a criminal record, 0 if the
tester does not. \\
\texttt{interact} \texttt{city} & 1 if tester interacted with employer
during the job application, 0 if tester does not interact with employer.
1 is job is located in the city center, 0 if job is located in the
suburbs. \\
\texttt{distance} & Job's average distance to downtown. \\
\texttt{custserv} & 1 if job is in the costumer service sector, 0 if it
is not. \\
\texttt{manualskill} & 1 if job requires manual skills, 0 if it does
not. \\
\bottomrule
\end{longtable}

For our classification exercise, we will consider \texttt{callback} the
label of our dependent variable; all other variables are our explanatory
variables.

\hypertarget{knn}{%
\subsection{KNN}\label{knn}}

\hypertarget{read-in-data-and-take-a-look-at-it}{%
\subsubsection{Read in data and take a look at
it:}\label{read-in-data-and-take-a-look-at-it}}

Read in the data, and omit NAs. You should end up with 694 observations.

\begin{Shaded}
\begin{Highlighting}[]
\FunctionTok{library}\NormalTok{(tidyverse)}
\end{Highlighting}
\end{Shaded}

\begin{verbatim}
## -- Attaching packages --------------------------------------- tidyverse 1.3.1 --
\end{verbatim}

\begin{verbatim}
## v ggplot2 3.3.5     v purrr   0.3.4
## v tibble  3.1.6     v dplyr   1.0.7
## v tidyr   1.1.4     v stringr 1.4.0
## v readr   2.1.1     v forcats 0.5.1
\end{verbatim}

\begin{verbatim}
## -- Conflicts ------------------------------------------ tidyverse_conflicts() --
## x dplyr::filter() masks stats::filter()
## x dplyr::lag()    masks stats::lag()
\end{verbatim}

\begin{Shaded}
\begin{Highlighting}[]
\NormalTok{data }\OtherTok{\textless{}{-}} \FunctionTok{read\_csv}\NormalTok{(}\StringTok{"criminalrecord.csv"}\NormalTok{) }\SpecialCharTok{\%\textgreater{}\%}
  \FunctionTok{drop\_na}\NormalTok{()}
\end{Highlighting}
\end{Shaded}

\begin{verbatim}
## Rows: 696 Columns: 9
\end{verbatim}

\begin{verbatim}
## -- Column specification --------------------------------------------------------
## Delimiter: ","
## dbl (9): jobid, callback, black, crimrec, interact, city, distance, custserv...
\end{verbatim}

\begin{verbatim}
## 
## i Use `spec()` to retrieve the full column specification for this data.
## i Specify the column types or set `show_col_types = FALSE` to quiet this message.
\end{verbatim}

\hypertarget{question-1-create-trainingtesting-data-split-in-r.}{%
\subsubsection{Question 1: Create Training/Testing data split in
R.}\label{question-1-create-trainingtesting-data-split-in-r.}}

\begin{enumerate}
\item Set the seed to 2019.
\item Randomly sample 70\% of the data for training and set aside the remaining 30\% for testing.
\item Create a function that calculates the Euclidean distance between 2 points. As a reminder, the formula for Euclidean distance is:
\begin{equation}
d(x_{1},x_{2})=\sqrt{\sum_{p=0}^n (x_{1p}-x_{2p})^2}
\end{equation}
\end{enumerate}

\begin{Shaded}
\begin{Highlighting}[]
\FunctionTok{set.seed}\NormalTok{(}\DecValTok{2019}\NormalTok{)}
\NormalTok{samp }\OtherTok{\textless{}{-}} \FunctionTok{sample\_n}\NormalTok{(data, }\DecValTok{70}\NormalTok{)}
\end{Highlighting}
\end{Shaded}

\hypertarget{question-2-knn-prediction-function}{%
\subsubsection{Question 2: KNN prediction
function}\label{question-2-knn-prediction-function}}

Write a function \texttt{knn} with the following characteristics:

\begin{itemize}
\item 3 arguments: test data, train data, and a value for k
\item Loops over all records of test and train data
\item Returns predicted class labels of test data
\end{itemize}

\hypertarget{question-3-accuracy-calculation}{%
\subsubsection{Question 3 Accuracy
calculation}\label{question-3-accuracy-calculation}}

Create a function called \texttt{accuracy} that calculates how accurate
your predictions are for the test labels. The accuracy function should
calculate the ratio of the number of correctly predicted labels to the
total number of predicted labels.

\hypertarget{question-4-make-your-prediction-find-your-accuracy}{%
\subsubsection{Question 4 Make your prediction, find your
accuracy}\label{question-4-make-your-prediction-find-your-accuracy}}

Using K=5, predict your labels for the testing data using the
\texttt{knn} function you wrote.

Append the prediction vector a column in your test dataframe and then
using the \texttt{accuracy()} method you wrote up in the previous
question print the accuracy of your KNN model. What is the accuracy rate
of your classification?

\end{document}
